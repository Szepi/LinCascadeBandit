%!TEX root = Paper.tex

\section{Background}
\label{sec:background}

Web pages in a search engine can be ranked automatically by fitting a model of user behavior in web search from historical click data \cite{radlinski05query,agichtein06improving}. The user is typically assumed to scan a list of $K$ web pages $A = (a_1, \dots, a_K)$, which we call \emph{items}. The items belong to some \emph{ground set} $E = \set{1, \dots, L}$, such as the set of all web pages. Many models of user behavior in web search exist \cite{becker07modeling,craswell08experimental,richardson07predicting}. Each of them explains the clicks of the user differently. We focus on the cascade model.

The \emph{cascade model} is a popular model of user behavior in web search \cite{craswell08experimental}. In this model, the user scans a list of $K$ items $A = (a_1, \dots, a_K) \in \Pi_K(E)$ from the first item $a_1$ to the last $a_K$, where $\Pi_K(E)$ is the set of all \emph{$K$-permutations} of set $E$. The model is parameterized by \emph{attraction probabilities} $\bar{w} \in [0, 1]^E$. After the user examines item $a_k$, the item attracts the user with probability $\bar{w}(a_k)$, \emph{independently} of the other items. If the user is attracted by item $a_k$, the user clicks on it and does not examine the remaining items. If the user is not attracted by item $a_k$, the user examines item $a_{k + 1}$. It is easy to see that the probability that item $a_k$ is examined is $\prod_{i = 1}^{k - 1} (1 - \bar{w}(a_i))$, and that the probability that at least one item in $A$ is attractive is $1 - \prod_{i = 1}^{K} (1 - \bar{w}({a_i}))$. This objective is maximized by $K$ most attractive items.

The cascade model assumes that the user clicks on at most one item. In practice, the user may click on multiple items. The cascade model cannot explain this pattern. Therefore, the model was extended in several directions, for instance to take into account \emph{multiple clicks} and the \emph{persistence} of users \cite{chapelle09dynamic,guo09click,guo09efficient}. The extended models explain click data better than the cascade model. Nevertheless, the cascade model is still very attractive, because it is simpler and can be reasonably fit to click data. Therefore, as a first step towards understanding more complex models, we study an online variant of the cascade model in this work.
